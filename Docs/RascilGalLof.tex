\documentclass[english]{article}
 
\usepackage{minted}
\usepackage{graphicx}
\usepackage{hyperref}

\begin{document}

\title{ RASCIL Galahad and IRIS - CL and job submission}

\maketitle  



This documentation follows the steps of the links below:
\begin{itemize}
    \item \footnote{\url{https://ska-telescope.gitlab.io/rascil/installation/RASCIL\_docker.html\#running-on-existing-docker-images}}RASCIL\_docker
    
    \item \footnote{\url{https://ska-telescope.gitlab.io/rascil/installation/RASCIL\_docker.html\#singularity}} RASCIL\_singularity
    
    \item \footnote{\url{https://gitlab.com/ska-telescope/rascil}} RASCIL\_gitlab

\end{itemize}

\section{RASCIL on Galahad and IRIS (CL):}
\begin{itemize}
    \item Set up environment variables (Galahad) under working directory:
\begin{minted}{python}
[<your-user>@galahad ~]$ ls /share/nas/<your-user>/
[<your-user>@galahad ~]$ mkdir /share/nas/<your-user>/.singularity
[<your-user>@galahad ~]$ export SINGULARITY_CACHEDIR=/share/nas/<youruser>/.singularity

\end{minted}

    \item Running on existing docker images
    
The RASCIL Dockerfiles are in a separate repository at https://gitlab.com/ska-telescope/rascil-docker.

The docker images for RASCIL are on nexus.engageska-portugal.pt at:
\begin{minted}{python}
nexus.engageska-portugal.pt/rascil-docker/rascil-base
nexus.engageska-portugal.pt/rascil-docker/rascil-full
\end{minted}
The first does not have the RASCIL test data but is smaller in size (2GB vs 4GB). However, for many of the tests and demonstrations the test data is needed.

    \item Pull the Rascil image:
 \begin{minted}{python}   
[<your-user>@galahad ~]$ singularity pull RASCIL-full.img 
docker://nexus.engageska-portugal.pt/rascil-docker/rascil-full
[<your-user>@galahad ~]$ singularity pull RASCIL-base.img 
docker://nexus.engageska-portugal.pt/rascil-docker/rascil-base
\end{minted}


    \item Running notebooks
    \begin{minted}{python}
[<your-user>@galahad ~]$ singularity exec RASCIL-full.img jupyter 
notebook --no-browser --ip 0.0.0.0  /rascil/examples/notebooks/
[I 10:51:44.514 NotebookApp] Serving notebooks from local directory:
/rascil/examples/notebooks
[I 10:51:44.514 NotebookApp] Jupyter Notebook 6.1.4 is running at:
[I 10:51:44.514 NotebookApp] http://galahad.ast.man.ac.uk:8888/
?token=26b1523066c7363b5575dde53d5d7780338bf3dc9cbe2102
[I 10:51:44.514 NotebookApp]  or http://127.0.0.1:8888/
?token=26b1523066c7363b5575dde53d5d7780338bf3dc9cbe2102
[I 10:51:44.514 NotebookApp] Use Control-C to stop this server and shut down all
kernels (twice to skip confirmation).
[C 10:51:44.519 NotebookApp]

To access the notebook, open this file in a browser:
    file:///home/<your-user>/.local/share/jupyter/runtime/nbserver-21541-open.html
Or copy and paste one of these URLs:
    http://galahad.ast.man.ac.uk:8888/
    ?token=26b1523066c7363b5575dde53d5d7780338bf3dc9cbe2102
    or http://127.0.0.1:8888/
    ?token=26b1523066c7363b5575dde53d5d7780338bf3dc9cbe2102
[I 10:51:56.498 NotebookApp] 302 GET 
/?token=26b1523066c7363b5575dde53d5d7780338bf3dc9cbe2102 
(10.242.203.134) 1.04ms


Access the notebooks on browser using http://galahad.ast.man.ac.uk:8888/
?token=26b1523066c7363b5575dde53d5d7780338bf3dc9cbe2102

Use CTRL <C> to shut down notebook server

\end{minted}

    \item Running RASCIL as a cluster:
    \begin{minted}{python}
[<your-user>@galahad ~]$ singularity exec RASCIL-full.img 
python3 /rascil/cluster_tests/ritoy/cluster_test_ritoy.py

Creating scheduler and 4 workers
<Client: 'tcp://127.0.0.1:46212' processes=4 threads=4, memory=67.34 GB>
53870592.0
*** Successfully reached end in 26.5 seconds ***

Note: use VNCViewer (see Appendix) to access links on Galahad, like Diagnostics page.
\end{minted}

    \item Running example script:
    \begin{minted}{python}
[<your-user>@galahad ~]$ singularity exec RASCIL-full.img python3 
/rascil/examples/scripts/imaging.py

creates 3 images output
[<your-user>@galahad ~]$ ls
 imaging_dirty.fits  imaging_psf.fits  imaging_restored.fits

 \end{minted}

\end{itemize}

\section{Job submission Galahad} 
\begin{minted}{python}
[<your-user>@galahad ~]$ cat  slrascil1.sh
#!/bin/bash
#SBATCH --ntasks 1
#SBATCH --time 5:0
#SBATCH --output=test_%j.log
pwd; hostname; date

module load python37base gcc920
CMD="singularity exec /home/<your-user>/RASCIL-full.img python3 
/rascil/examples/scripts/imaging.py"
eval $CMD

[<your-user>@galahad ~]$  sbatch slrascil1.sh
Submitted batch job 3404


[<your-user>@galahad ~]$  squeue
JOBID PARTITION     NAME     USER ST       TIME  NODES NODELIST(REASON)
3404   CLUSTER slrascil   <your-user>R       0:18      1 compute-0-7

 \end{minted}


\section{Job submission IRIS}
From the server where dirac is installed:
\begin{itemize}
    \item start proxy before using any dms commands
    \begin{minted}{python}
    bash-4.2$ source bashrc
    bash-4.2$ dirac-proxy-init -g skatelescope.eu_user -M
    \end{minted}

 \item Add the RASCIL container to the filecathalog using command "dirac-dms-add-file"  
  \begin{minted}{python}
dirac-dms-add-file LFN:/skatelescope.eu/user/c/<your-user>/rascil/RASCIL-full.img  
RASCIL-full.img  UKI-NORTHGRID-MAN-HEP-disk
\end{minted}

\item check where the files has been uploaded using command "dirac-dms-filecatalog-cli"

\end{itemize}


\subsection{Job submission - submit .jdl }
\begin{itemize}
    \item create .jdl and .sh files

  \begin{minted}{python}

cat simpleR1.jdl
JobName = "InputAndOuputSandbox";
Executable = "testR1.sh";
StdOutput = "StdOut";
StdError = "StdErr";
InputSandbox = {"testR1.sh"};
InputData = {"LFN:/skatelescope.eu/user/c/<your-user>/rascil/RASCIL-full.img"};
OutputSandbox = {"StdOut","StdErr"};
OutputData={"imaging_dirty.fits","imaging_psf.fits","imaging_restored.fits"};
OutputSE ="UKI-NORTHGRID-MAN-HEP-disk";
Site = "LCG.UKI-NORTHGRID-MAN-HEP.uk";


cat testR1.sh
#!/bin/bash
singularity exec --cleanenv -H $PWD:/srv --pwd /srv -C RASCIL-full.img
python3 /rascil/examples/scripts/imaging.py;

\end{minted}

\item Submit the job
\begin{minted}{python}

bash-4.2$ dirac-wms-job-submit simpleR1.jdl
JobID = 25260750

bash-4.2$ dirac-wms-job-status 25260750
JobID=25260750 Status=Running; MinorStatus=Input Data Resolution; 
Site=LCG.UKINORTHGRID-MAN-HEP.uk;

bash-4.2$ dirac-wms-job-status 25260750
JobID=25260750 Status=Done; MinorStatus=Execution Complete; 
Site=LCG.UKINORTHGRID-MAN-HEP.uk;
\end{minted}

\item Get output data and output file
\begin{minted}{python}

bash-4.2$ dirac-wms-job-get-output-data 25336768
Job 25336768 output data retrieved
bash-4.2$ ls
-rw-r--r--. 1 <your-user> users6 2102400 May 14 17:32 imaging_dirty.fits
-rw-r--r--. 1 <your-user> users6 2102400 May 14 17:32 imaging_psf.fits
-rw-r--r--. 1 <your-user> users6 2102400 May 14 17:32 imaging_restored.fits

bash-4.2$ dirac-wms-job-get-output 25336768
Job output sandbox retrieved in
/raid/scratch/<your-user>/dirac_ui/tests/rascilTests/ 25336768/
bash-4.2$ cd 25336768
bash-4.2$ ls
StdErr StdOut
bash-4.2$ cat StdErr
INFO: Convert SIF file to sandbox...
INFO: Cleaning up image...

\end{minted}

\end{itemize}


\subsection{Job submission - submit .py}

\begin{itemize}
    \item Set up environment variables:
   \begin{minted}{python}
   
#SET THE PATH PYTHON 2.7 INTO $PATH
#PATH to python 2.7 added
eg bash-4.2$ export PATH=/usr/local/casa/bin/python:$PATH

\end{minted}



\item the job to be submitted and the .sh script
  \begin{minted}{python}

bash-4.2$ cat jobpy.py
import os
import sys
import time
# setup DIRAC
from DIRAC.Core.Base import Script
Script.parseCommandLine(ignoreErrors=False)
from DIRAC.Interfaces.API.Job import Job
from DIRAC.Interfaces.API.Dirac import Dirac
from DIRAC.Core.Security.ProxyInfo import getProxyInfo
SitesList = ['LCG.UKI-NORTHGRID-MAN-HEP.uk']
SEList = ['UKI-NORTHGRID-MAN-HEP-disk']
dirac = Dirac()
j = Job(stdout='StdOut', stderr='StdErr')
j.setName('TestJob')
j.setInputSandbox(["testR1py.sh"])
j.setInputData(['LFN:/skatelescope.eu/user/c/<your-user>/rascil/RASCILfull.img'])
j.setOutputSandbox(['StdOut','StdErr'])
j.setOutputData(['imaging_dirty.fits','imaging_psf.fits','imaging_restored.fits'],
outputSE='UKI-NORTHGRID-MAN-HEP-disk')
j.setExecutable('testR1py.sh')
jobID = dirac.submitJob(j)
print 'Submission Result: ', jobID


bash-4.2$ cat testR1py.sh
#!/bin/bash
singularity exec --cleanenv -H $PWD:/srv --pwd /srv -C RASCIL-full1.img
python3 /rascil/examples/scripts/imaging.py
\end{minted}

\item Submitting the job
  \begin{minted}{python}

bash-4.2$ python jobpy.py
Submission Result: {'requireProxyUpload': False, 'OK': True, 'rpcStub':
(('WorkloadManagement/JobManag er', {'delegatedDN':
None, 'timeout': 600, 'skipCACheck': False, 'keepAliveLapse': 150,
'delegatedGroup ': None}), 'submitJob', ('[ \n
Origin = DIRAC;\n Executable = "$DIRACROOT/scripts/dirac-jobexec";
\n StdError = StdErr;\n LogLevel = info;\n OutputSE = UKI-NORTHGRIDMAN-
HEP-disk;\n InputSa ndbox = \n {\n
"testR1py.sh",\n "SB:GridPPSandboxSE|/SandBox/i/iulia.c.cim
pan.skatelescope.eu_user/cf8/ca6/cf8ca689995e24c01c068eb6f34126b8.tar.bz2"\n
};\n JobName = T estJob;\n Priority = 1;\n
Arguments = "jobDescription.xml -o LogLevel=info";\n JobGroup = skat
elescope.eu;\n OutputSandbox = \n {\n StdOut,\n
StdErr,\n Sc ript1_testR1py.sh.log\n
};\n StdOutput = StdOut;\n InputData = LFN:/skatelescope.eu/user/c
/<your-user>/rascil/RASCIL-full1.img;\n JobType = User;\n OutputData = \n
{\n imagin g_dirty.fits,\n
imaging_psf.fits,\n imaging_restored.fits\n };\n]',)), 'Va
lue': 25344748, 'JobID': 25344748}
\end{minted}

\item Get the results
  \begin{minted}{python}

bash-4.2$ dirac-wms-job-get-output 25344748
Job output sandbox retrieved in 
/raid/scratch/<your-user>/dirac_ui/tests/rascilTests/25344748/

bash-4.2$ cd 25344748
bash-4.2$ ls
Script1_testR1py.sh.log StdOut

bash-4.2$ dirac-wms-job-get-output-data 25344748
Job 25344748 output data retrieved
bash-4.2$ ls
imaging_dirty.fits imaging_psf.fits imaging_restored.fits
Script1_testR1py.sh.log StdOut
\end{minted}

\end{itemize}

\vspace{7 cm}

\section{Appendix}
 \begin{minted}{python}
You run vncserver on galahad (already installed). On your windows PC use:
https://www.tightvnc.com/download-old.php as your vnc viewer.

When you run vncserver for the first time you will set up a password. 
It will report it has created a virtual display galahad.ast.man.ac.uk:X
The X will be a number. You then use that address in your vnc viewer

[<your-user>@galahad ~]$ vncserver
[<your-user>@galahad ~]$ vncserver -kill :3
Killing Xvnc process ID 35841
\end{minted}



With vnc I would suggest editing the default .vnc/xstartup file 
(created after you run vncserver for the first time) to change the last line to run /usr/bin/icewm as the window manager rather than xinitrc.
You should then kill off your first vncserver and run it again to pick up the change.
This avoids a bug where sometimes the VNC just displays a black screen.

\begin{minted}{python}

[<your-user>@galahad ~]$ cat .vnc/xstartup
#!/bin/shunset SESSION_MANAGER
unset DBUS_SESSION_BUS_ADDRESS
#exec /etc/X11/xinit/xinitrc
/usr/bin/icewm
[<your-user>@galahad ~]$ vncserver #restarting the server
\end{minted}
How to find the host for the for the diagnostics page? It would be whichever host has started it, so
use squeue to see what host is running your job and then it would be for example http://compute-0-5:8787


\begin{minted}{python}
[<your-user>@galahad ~]$ squeue

\end{minted}



\end{document}
